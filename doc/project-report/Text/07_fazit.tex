\section{Fazit}

    \subsection{Learnings}
        Die wichtigste Erkenntnis aus der Arbeit, ist die steile Lernkurve. Aspire ist umfangreich und deckt deckt alle möglichen Aspekte einer verteilten Service Landschaft ab. Eine Semesterarbeit ist ein viel zu kleiner Rahmen, um die Gänze von Aspire zu erfassen. Das Kickstarter Template deckt aber eben eine sehr breite Palette von Aspekten ab. Das muss es auch, damit der Setup kurz ist und dem Entwickler viel Aufbauarbeit abnimmt. Diese Breite des Templates müsste auch in einer Aspire Integration abgedeckt werden und das war schlussendlich auch der Hauptgrund, dass die Integration von Aspire in das Kickstarter Template nicht weiter verfolgt wurde. Sondern ein einfacher Prototyp mit einem begrenzten Umfang realisiert wurde, so dass das bereits gewonnene Know-How ausreichend war, um den Prototypen aufzubauen.

        Ein weiteres wichtiges Learning ist, dass Aspire für die aktuellen Anwendungen des Kickstarter Templates, keinen Mehrwert bietet. Bei Isolutions werden selten Applikationen entwickelt, welche vom Umfang von Aspire profitieren würden. Das kommt vor allem daher, dass viele Kunden nicht explizit Cloud-Native Applikationen verlangen und von den Anforderungen her auch nicht benötigen. Für die Zielgruppe von Isolutions ist das Kickstarter Template heute vollkommen ausreichend. Deshalb wäre der nötige Aufwand für die Integration von Aspire nicht gerechtfertigt.

    \subsection{Persönliche Betrachtung}
        Aspire .Net verspricht viele sehr interessante Verbesserungen für einen Entwickler. Insbesondere für die Entwicklung von Applikationen für die Azure Cloud. Das Erlebnis als Entwickler eine Aspire App aufzubauen, zu Deployen und auf Azure im Betrieb zu sehen, macht sehr viel Spass. Aspire .Net in eine bestehende App zu integrieren allerdings nicht. Jedenfalls nicht mit dem aktuellen Kenntnisstand. Hier ist Know-How und Verständnis des Aspire Stacks der Schlüssel zum Erfolg und dieses Know-How muss man sich heute noch in grossen Teilen durch Try and Error aneignen. Im Rahmen eines Semesters ist dieser Know-How Aufbau nicht möglich. Da Aspire .Net ein ganzer Stack ist und ganze Service Landschaften abbilden soll, ist auch das nötige Wissen sehr breit. Verschiedene Technologien, Backend, Frontend, Deployment, IaC, Cloud Native, usw. sind alles Begriffe welche Aspire ausmachen und entsprechendes Know-How voraussetzen, um gute und sichere Lösungen mit Aspire zu realisieren. Aspire setzt die Bereitschaft voraus, breites Know-How aufzubauen. Kann dieses Wissen gemeistert werden und werden die Kinderkrankheiten von Aspire behoben, kann sich der Stack zu einem sehr interessanten Werkzeug entwickeln, für die Realisierung von Cloud-Native Applikationen auf Azure.

    \subsection{Ausblicke}
        Seit ein paar Monaten gilt Aspire .Net als Production-Ready und produktive Lösungen sind im Einsatz. Trotzdem gibt es noch Bugs und Probleme im Aspire .Net Stack selber, welche behoben werden müssen. In Zukunft kann der Einsatz von Aspire auch bei Isolutions interessant werden, insbesondere wenn vermehrt Cloud Services eingesetzt werden können. Bis dahin wird sich auch die Knowledgebase im Internet vergrössern und die Implementierung von Aspire .Net erleichtern. Isolutions Entwickler und auch Ich persönlich setzen sich auf freiwilliger Basis heute mit Aspire auseinander und bauen internes Know-How auf. Rechtfertigt sich der Einsatz von Aspire, würde ich Aspire ohne bedenken einsetzen.
        
