\section{Implementierung Aspire .NET Kickstartertemplate}
    In dieser Sektion werden die Arbeiten und Ergebnisse in den Iterationen beschreiben.

    \subsection{Iteration 1: Know-How Aufbau}
        Das gewonnene Know-How über Aspire .Net ist im Analyse Abschnitt dieses Berichtes beschrieben. In diesem Abschnitt wird für die arbeit bessonders notwendiges Know-How dokumentiert.

        Ein Aspire .Net Projekt besteht, wie bereits beschreiben, aus mindestens drei Projekten. Dem AppHost, ServiceDefaults und einem Service mit individueller Applikationslogik. AppHost und ServiceDefaults sollen hier genauer beschreiben werden.

        \subsubsection{AppHost}
            Das AppHost Projekt ist das Haupt- und Startprojekt einer Aspire .Net Solution. Es enthält die Servicekomposition und ist verantwortlich für deren Orchestrierung.
            \begin{minted}{python3}
            var postgres = builder.AddPostgres("postgres").PublishAsAzurePostgresFlexibleServer();
            var postgresdb = postgres.AddDatabase("postgresdb");

            var api = builder.AddProject<MinimalApi>("messagesapi")
                .WithReference(postgresdb)
                .WithExternalHttpEndpoints();
            \end{minted}
            Im Beispiel

    \subsubsection{AppHost}


    \subsection{Iteration 1: Erkenntisse}

    \subsection{Iteration 2: Umbau des Kickstarter Templates}

    \subsection{Iteration 2: Erkenntnisse}

    \subsection{Iteration 3: Aufbau eines neuen Protoypes}

    \subsection{Iteration 3: Erkenntnisse}



