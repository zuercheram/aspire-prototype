% in Anlehnung an https://github.com/derdanu/akad-vorlage
% Autor: Lukas Lück
%-------------------------------------------
% Vorgaben Assignment aus Studienheft SQL301
%-------------------------------------------
% Umfang: 8 - 10 Seiten (inkl. Abbildungen und Tabellen, ohne Deckblatt, Gliederung und Literaturverzeichnis, Eidesstattliche Erklaerung)
% Zeilenabstand: 1,5, Tabellen 1,0 
% Schriftart: frei
% Schriftgrad: 11 oder 12 pt
% Tabellen auch 10pt
% Silbentrennung ok, auf "Missratene" Trennung achten!
% Variablen, physikalische Groessen und Funktionszeichen werden kursiv gedruckt.
% Ränder: links: 4,5 cm, rechts 2,0 cm, oben und unten jeweils 3,0 cm
% Deckblatt: (Name, Adresse, AKAD-E-Mail-Adresse, Immatrikulationsnummer, Modulbezeichnung, Thema, Abgabedatum, Dozent)

% Reihenfolge (siehe SQL301, S. 71):
% Titelblatt, ohne Nummerierung
% Inhaltsverzeichnis (ohne Nummerierung) -> Abbildungsverzeichnis (Nummerierung in Römisch, beginnend mit "III")-> Tabellenverzeichnis -> Abk.-Verzeichniss
% Textteil, Nummerierung in arabisch, beginnen mit "1"
% Anhang -> Literaturverzeichnis, Nummerierung in Römisch, folgend auf Römische Nummerierung von Verzeichnissen
\documentclass[listof=totoc, bibliography=totoc, a4paper, 12pt, numbers=noenddot]{scrartcl}
\usepackage[ngerman]{babel} % Sprachpaket
\usepackage{newtxtext,newtxmath} % Konsistente Text- und Mathematikschrift
\usepackage[onehalfspacing]{setspace} % Zeilenabstand
\usepackage[]{geometry} % Für Anpassung Seitenränder
\usepackage[T1]{fontenc} % Bessere Trennung (kann entfernt werden, wenn nicht benötigt)
\usepackage[utf8]{inputenc} % UTF-8 Unterstützung (kann entfernt werden, wenn nicht benötigt)
\usepackage{microtype} % "Entspanntere" Umbrüche

% Pakete für Tabellen
\usepackage{multirow}
\usepackage{longtable}
\usepackage{vcell}
\usepackage{colortbl}
\usepackage{hhline}
\usepackage{float} % Notwendig für figure[h]
\usepackage[table]{xcolor}

% Pakete für Mathematik
\usepackage{amsmath} % Erweiterte mathematische Funktionen
\usepackage{mathtools} % Erweiterung für amsmath
\usepackage{amsfonts} % Zahlenmengen Zeichen
\usepackage{bm} % unterstützt bold in Formeln

% Pakete für Grafiken und Beschriftungen
\usepackage{graphicx}
\usepackage{caption}
\usepackage{subcaption}
\usepackage{tikz}
\usepackage[export]{adjustbox}

% Pakete für Kopf- und Fußzeilen
\usepackage[headsepline=1pt,plainheadsepline, singlespacing=true]{scrlayer-scrpage}         % Erstellung von selbst definierten Kopfzeilen
\clearpairofpagestyles%\clearscrheadfoot veraltet
\ohead*{\pagemark}
\ihead{\headmark}
\setkomafont{pageheadfoot}{\normalfont}

% Pakete für PDFs
\usepackage{pdfpages} % Einbinden von PDF-Dateien
\usepackage{pdflscape}
\usepackage{rotating}
\usepackage{lscape}

% Paket für Abkürzungsverzeichnis
\usepackage{acro}
\acsetup{make-links} % Setup linking with hyperref

% Weitere Einstellungen
\usepackage{url}
\usepackage[stable]{footmisc}
%\usepackage{tocloft} %Zeilenabstand in Verzeichnissen anpassen
%\usepackage{hyphenat} %Silbentrennung für einzelne Wörter verhindern mit \nohyphens{text}

% Pakete für Literatur + zitieren, apa
\usepackage[backend=biber, style=numeric, autocite=footnote, maxnames=99]{biblatex}
\addbibresource{doc/project-report/Verzeichnisse/literatur.bib}
\usepackage{csquotes} % für Zitate

%% Für Codeblöcke mit Syntax-Highlighting
%% http://www.ctan.org/tex-archive/macros/latex/contrib/minted/
%% Einkommentieren fuer Minted Syntax Highlighting
%\usepackage{minted}
%\definecolor{bg}{rgb}{0.95,0.95,0.95}
\newcommand*{\docType}{Assignment}
% Titel
\newcommand*{\docTitle}{UML} 
% Betreff
\newcommand*{\betreff}{Objektorientierte Software-Entwicklung, UML} 
% Modul
\newcommand*{\modul}{SWE24}
% Betreuer
\newcommand*{\dozent}{Max Musterman } 
% Vor- und Nachname
\newcommand*{\name}{Dein, Name}
% Straße und Hausnummer
\newcommand*{\strasse}{Staße Nr. 01} 
% Plz und Ort
\newcommand*{\plzort}{12345 Ort} 
% Immatrikulationsnummer
\newcommand*{\immanr}{234567}
%Abgabedatum, z.B. 10. Juni 2024 oder \today
\newcommand*{\finalDate}{\today}
%Studiengangsbezeichnung
\newcommand*{\studiFieldOne}{Informatik - Master}
\newcommand*{\studiFieldTwo}{of Science (M. Sc.)}
%%E-Mail
\newcommand*{\mail}{email.adresse@provider.de}
%pdf-KeyWords
\newcommand*{\pdfKeyWords}{Keywords, für, das pdf, Dokument}
%pdf-Description
\newcommand*{\pdfDescription}{Assignment zum Thema UML}

% %%%%%%%%%%%%%%%%%%%%%%%%%%%%%%%%%%%%%%%%%%%%%%%%%%%%%%%%%%%%%%%%%%%%%%%%%%%%%%%% 
%%% ~ Arduino Language - Arduino IDE Colors ~                                  %%%
%%%                                                                            %%%
%%% Kyle Rocha-Brownell | 10/2/2017 | No Licence                               %%%
%%% -------------------------------------------------------------------------- %%%
%%%                                                                            %%%
%%% Place this file in your working directory (next to the latex file you're   %%%
%%% working on).  To add it to your project, place:                            %%%
%%%     %%%%%%%%%%%%%%%%%%%%%%%%%%%%%%%%%%%%%%%%%%%%%%%%%%%%%%%%%%%%%%%%%%%%%%%%%%%%%%%% 
%%% ~ Arduino Language - Arduino IDE Colors ~                                  %%%
%%%                                                                            %%%
%%% Kyle Rocha-Brownell | 10/2/2017 | No Licence                               %%%
%%% -------------------------------------------------------------------------- %%%
%%%                                                                            %%%
%%% Place this file in your working directory (next to the latex file you're   %%%
%%% working on).  To add it to your project, place:                            %%%
%%%     %%%%%%%%%%%%%%%%%%%%%%%%%%%%%%%%%%%%%%%%%%%%%%%%%%%%%%%%%%%%%%%%%%%%%%%%%%%%%%%% 
%%% ~ Arduino Language - Arduino IDE Colors ~                                  %%%
%%%                                                                            %%%
%%% Kyle Rocha-Brownell | 10/2/2017 | No Licence                               %%%
%%% -------------------------------------------------------------------------- %%%
%%%                                                                            %%%
%%% Place this file in your working directory (next to the latex file you're   %%%
%%% working on).  To add it to your project, place:                            %%%
%%%    \input{arduinoLanguage.tex}                                             %%%
%%% somewhere before \begin{document} in your latex file.                      %%%
%%%                                                                            %%%
%%% In your document, place your arduino code between:                         %%%
%%%   \begin{lstlisting}[language=Arduino]                                     %%%
%%% and:                                                                       %%%
%%%   \end{lstlisting}                                                         %%%
%%%                                                                            %%%
%%% Or create your own style to add non-built-in functions and variables.      %%%
%%%                                                                            %%%
 %%%%%%%%%%%%%%%%%%%%%%%%%%%%%%%%%%%%%%%%%%%%%%%%%%%%%%%%%%%%%%%%%%%%%%%%%%%%%%%% 

\usepackage{color}
\usepackage{listings}    
\usepackage{courier}

%%% Define Custom IDE Colors %%%
\definecolor{arduinoGreen}    {rgb} {0.17, 0.43, 0.01}
\definecolor{arduinoGrey}     {rgb} {0.47, 0.47, 0.33}
\definecolor{arduinoOrange}   {rgb} {0.8 , 0.4 , 0   }
\definecolor{arduinoBlue}     {rgb} {0.01, 0.61, 0.98}
\definecolor{arduinoDarkBlue} {rgb} {0.0 , 0.2 , 0.5 }

%%% Define Arduino Language %%%
\lstdefinelanguage{Arduino}{
  language=C++, % begin with default C++ settings 
%
%
  %%% Keyword Color Group 1 %%%  (called KEYWORD3 by arduino)
  keywordstyle=\color{arduinoGreen},   
  deletekeywords={  % remove all arduino keywords that might be in c++
                break, case, override, final, continue, default, do, else, for, 
                if, return, goto, switch, throw, try, while, setup, loop, export, 
                not, or, and, xor, include, define, elif, else, error, if, ifdef, 
                ifndef, pragma, warning,
                HIGH, LOW, INPUT, INPUT_PULLUP, OUTPUT, DEC, BIN, HEX, OCT, PI, 
                HALF_PI, TWO_PI, LSBFIRST, MSBFIRST, CHANGE, FALLING, RISING, 
                DEFAULT, EXTERNAL, INTERNAL, INTERNAL1V1, INTERNAL2V56, LED_BUILTIN, 
                LED_BUILTIN_RX, LED_BUILTIN_TX, DIGITAL_MESSAGE, FIRMATA_STRING, 
                ANALOG_MESSAGE, REPORT_DIGITAL, REPORT_ANALOG, SET_PIN_MODE, 
                SYSTEM_RESET, SYSEX_START, auto, int8_t, int16_t, int32_t, int64_t, 
                uint8_t, uint16_t, uint32_t, uint64_t, char16_t, char32_t, operator, 
                enum, delete, bool, boolean, byte, char, const, false, float, double, 
                null, NULL, int, long, new, private, protected, public, short, 
                signed, static, volatile, String, void, true, unsigned, word, array, 
                sizeof, dynamic_cast, typedef, const_cast, struct, static_cast, union, 
                friend, extern, class, reinterpret_cast, register, explicit, inline, 
                _Bool, complex, _Complex, _Imaginary, atomic_bool, atomic_char, 
                atomic_schar, atomic_uchar, atomic_short, atomic_ushort, atomic_int, 
                atomic_uint, atomic_long, atomic_ulong, atomic_llong, atomic_ullong, 
                virtual, PROGMEM,
                Serial, Serial1, Serial2, Serial3, SerialUSB, Keyboard, Mouse,
                abs, acos, asin, atan, atan2, ceil, constrain, cos, degrees, exp, 
                floor, log, map, max, min, radians, random, randomSeed, round, sin, 
                sq, sqrt, tan, pow, bitRead, bitWrite, bitSet, bitClear, bit, 
                highByte, lowByte, analogReference, analogRead, 
                analogReadResolution, analogWrite, analogWriteResolution, 
                attachInterrupt, detachInterrupt, digitalPinToInterrupt, delay, 
                delayMicroseconds, digitalWrite, digitalRead, interrupts, millis, 
                micros, noInterrupts, noTone, pinMode, pulseIn, pulseInLong, shiftIn, 
                shiftOut, tone, yield, Stream, begin, end, peek, read, print, 
                println, available, availableForWrite, flush, setTimeout, find, 
                findUntil, parseInt, parseFloat, readBytes, readBytesUntil, readString, 
                readStringUntil, trim, toUpperCase, toLowerCase, charAt, compareTo, 
                concat, endsWith, startsWith, equals, equalsIgnoreCase, getBytes, 
                indexOf, lastIndexOf, length, replace, setCharAt, substring, 
                toCharArray, toInt, press, release, releaseAll, accept, click, move, 
                isPressed, isAlphaNumeric, isAlpha, isAscii, isWhitespace, isControl, 
                isDigit, isGraph, isLowerCase, isPrintable, isPunct, isSpace, 
                isUpperCase, isHexadecimalDigit, 
                }, 
  morekeywords={   % add arduino structures to group 1
                break, case, override, final, continue, default, do, else, for, 
                if, return, goto, switch, throw, try, while, setup, loop, export, 
                not, or, and, xor, include, define, elif, else, error, if, ifdef, 
                ifndef, pragma, warning,
                }, 
% 
%
  %%% Keyword Color Group 2 %%%  (called LITERAL1 by arduino)
  keywordstyle=[2]\color{arduinoBlue},   
  keywords=[2]{   % add variables and dataTypes as 2nd group  
                HIGH, LOW, INPUT, INPUT_PULLUP, OUTPUT, DEC, BIN, HEX, OCT, PI, 
                HALF_PI, TWO_PI, LSBFIRST, MSBFIRST, CHANGE, FALLING, RISING, 
                DEFAULT, EXTERNAL, INTERNAL, INTERNAL1V1, INTERNAL2V56, LED_BUILTIN, 
                LED_BUILTIN_RX, LED_BUILTIN_TX, DIGITAL_MESSAGE, FIRMATA_STRING, 
                ANALOG_MESSAGE, REPORT_DIGITAL, REPORT_ANALOG, SET_PIN_MODE, 
                SYSTEM_RESET, SYSEX_START, auto, int8_t, int16_t, int32_t, int64_t, 
                uint8_t, uint16_t, uint32_t, uint64_t, char16_t, char32_t, operator, 
                enum, delete, bool, boolean, byte, char, const, false, float, double, 
                null, NULL, int, long, new, private, protected, public, short, 
                signed, static, volatile, String, void, true, unsigned, word, array, 
                sizeof, dynamic_cast, typedef, const_cast, struct, static_cast, union, 
                friend, extern, class, reinterpret_cast, register, explicit, inline, 
                _Bool, complex, _Complex, _Imaginary, atomic_bool, atomic_char, 
                atomic_schar, atomic_uchar, atomic_short, atomic_ushort, atomic_int, 
                atomic_uint, atomic_long, atomic_ulong, atomic_llong, atomic_ullong, 
                virtual, PROGMEM,
                },  
% 
%
  %%% Keyword Color Group 3 %%%  (called KEYWORD1 by arduino)
  keywordstyle=[3]\bfseries\color{arduinoOrange},
  keywords=[3]{  % add built-in functions as a 3rd group
                Serial, Serial1, Serial2, Serial3, SerialUSB, Keyboard, Mouse,
                },      
%
%
  %%% Keyword Color Group 4 %%%  (called KEYWORD2 by arduino)
  keywordstyle=[4]\color{arduinoOrange},
  keywords=[4]{  % add more built-in functions as a 4th group
                abs, acos, asin, atan, atan2, ceil, constrain, cos, degrees, exp, 
                floor, log, map, max, min, radians, random, randomSeed, round, sin, 
                sq, sqrt, tan, pow, bitRead, bitWrite, bitSet, bitClear, bit, 
                highByte, lowByte, analogReference, analogRead, 
                analogReadResolution, analogWrite, analogWriteResolution, 
                attachInterrupt, detachInterrupt, digitalPinToInterrupt, delay, 
                delayMicroseconds, digitalWrite, digitalRead, interrupts, millis, 
                micros, noInterrupts, noTone, pinMode, pulseIn, pulseInLong, shiftIn, 
                shiftOut, tone, yield, Stream, begin, end, peek, read, print, 
                println, available, availableForWrite, flush, setTimeout, find, 
                findUntil, parseInt, parseFloat, readBytes, readBytesUntil, readString, 
                readStringUntil, trim, toUpperCase, toLowerCase, charAt, compareTo, 
                concat, endsWith, startsWith, equals, equalsIgnoreCase, getBytes, 
                indexOf, lastIndexOf, length, replace, setCharAt, substring, 
                toCharArray, toInt, press, release, releaseAll, accept, click, move, 
                isPressed, isAlphaNumeric, isAlpha, isAscii, isWhitespace, isControl, 
                isDigit, isGraph, isLowerCase, isPrintable, isPunct, isSpace, 
                isUpperCase, isHexadecimalDigit, 
                },      
%
%
  %%% Set Other Colors %%%
  stringstyle=\color{arduinoDarkBlue},    
  commentstyle=\color{arduinoGrey},    
%          
%   
  %%%% Line Numbering %%%%
  numbers=left,                    
  numbersep=5pt,                   
  numberstyle=\color{arduinoGrey},    
  %stepnumber=2,                      % show every 2 line numbers
%
%
  %%%% Code Box Style %%%%
  breaklines=true,                    % wordwrapping
  tabsize=2,         
  basicstyle=\ttfamily  
}
                                             %%%
%%% somewhere before \begin{document} in your latex file.                      %%%
%%%                                                                            %%%
%%% In your document, place your arduino code between:                         %%%
%%%   \begin{lstlisting}[language=Arduino]                                     %%%
%%% and:                                                                       %%%
%%%   \end{lstlisting}                                                         %%%
%%%                                                                            %%%
%%% Or create your own style to add non-built-in functions and variables.      %%%
%%%                                                                            %%%
 %%%%%%%%%%%%%%%%%%%%%%%%%%%%%%%%%%%%%%%%%%%%%%%%%%%%%%%%%%%%%%%%%%%%%%%%%%%%%%%% 

\usepackage{color}
\usepackage{listings}    
\usepackage{courier}

%%% Define Custom IDE Colors %%%
\definecolor{arduinoGreen}    {rgb} {0.17, 0.43, 0.01}
\definecolor{arduinoGrey}     {rgb} {0.47, 0.47, 0.33}
\definecolor{arduinoOrange}   {rgb} {0.8 , 0.4 , 0   }
\definecolor{arduinoBlue}     {rgb} {0.01, 0.61, 0.98}
\definecolor{arduinoDarkBlue} {rgb} {0.0 , 0.2 , 0.5 }

%%% Define Arduino Language %%%
\lstdefinelanguage{Arduino}{
  language=C++, % begin with default C++ settings 
%
%
  %%% Keyword Color Group 1 %%%  (called KEYWORD3 by arduino)
  keywordstyle=\color{arduinoGreen},   
  deletekeywords={  % remove all arduino keywords that might be in c++
                break, case, override, final, continue, default, do, else, for, 
                if, return, goto, switch, throw, try, while, setup, loop, export, 
                not, or, and, xor, include, define, elif, else, error, if, ifdef, 
                ifndef, pragma, warning,
                HIGH, LOW, INPUT, INPUT_PULLUP, OUTPUT, DEC, BIN, HEX, OCT, PI, 
                HALF_PI, TWO_PI, LSBFIRST, MSBFIRST, CHANGE, FALLING, RISING, 
                DEFAULT, EXTERNAL, INTERNAL, INTERNAL1V1, INTERNAL2V56, LED_BUILTIN, 
                LED_BUILTIN_RX, LED_BUILTIN_TX, DIGITAL_MESSAGE, FIRMATA_STRING, 
                ANALOG_MESSAGE, REPORT_DIGITAL, REPORT_ANALOG, SET_PIN_MODE, 
                SYSTEM_RESET, SYSEX_START, auto, int8_t, int16_t, int32_t, int64_t, 
                uint8_t, uint16_t, uint32_t, uint64_t, char16_t, char32_t, operator, 
                enum, delete, bool, boolean, byte, char, const, false, float, double, 
                null, NULL, int, long, new, private, protected, public, short, 
                signed, static, volatile, String, void, true, unsigned, word, array, 
                sizeof, dynamic_cast, typedef, const_cast, struct, static_cast, union, 
                friend, extern, class, reinterpret_cast, register, explicit, inline, 
                _Bool, complex, _Complex, _Imaginary, atomic_bool, atomic_char, 
                atomic_schar, atomic_uchar, atomic_short, atomic_ushort, atomic_int, 
                atomic_uint, atomic_long, atomic_ulong, atomic_llong, atomic_ullong, 
                virtual, PROGMEM,
                Serial, Serial1, Serial2, Serial3, SerialUSB, Keyboard, Mouse,
                abs, acos, asin, atan, atan2, ceil, constrain, cos, degrees, exp, 
                floor, log, map, max, min, radians, random, randomSeed, round, sin, 
                sq, sqrt, tan, pow, bitRead, bitWrite, bitSet, bitClear, bit, 
                highByte, lowByte, analogReference, analogRead, 
                analogReadResolution, analogWrite, analogWriteResolution, 
                attachInterrupt, detachInterrupt, digitalPinToInterrupt, delay, 
                delayMicroseconds, digitalWrite, digitalRead, interrupts, millis, 
                micros, noInterrupts, noTone, pinMode, pulseIn, pulseInLong, shiftIn, 
                shiftOut, tone, yield, Stream, begin, end, peek, read, print, 
                println, available, availableForWrite, flush, setTimeout, find, 
                findUntil, parseInt, parseFloat, readBytes, readBytesUntil, readString, 
                readStringUntil, trim, toUpperCase, toLowerCase, charAt, compareTo, 
                concat, endsWith, startsWith, equals, equalsIgnoreCase, getBytes, 
                indexOf, lastIndexOf, length, replace, setCharAt, substring, 
                toCharArray, toInt, press, release, releaseAll, accept, click, move, 
                isPressed, isAlphaNumeric, isAlpha, isAscii, isWhitespace, isControl, 
                isDigit, isGraph, isLowerCase, isPrintable, isPunct, isSpace, 
                isUpperCase, isHexadecimalDigit, 
                }, 
  morekeywords={   % add arduino structures to group 1
                break, case, override, final, continue, default, do, else, for, 
                if, return, goto, switch, throw, try, while, setup, loop, export, 
                not, or, and, xor, include, define, elif, else, error, if, ifdef, 
                ifndef, pragma, warning,
                }, 
% 
%
  %%% Keyword Color Group 2 %%%  (called LITERAL1 by arduino)
  keywordstyle=[2]\color{arduinoBlue},   
  keywords=[2]{   % add variables and dataTypes as 2nd group  
                HIGH, LOW, INPUT, INPUT_PULLUP, OUTPUT, DEC, BIN, HEX, OCT, PI, 
                HALF_PI, TWO_PI, LSBFIRST, MSBFIRST, CHANGE, FALLING, RISING, 
                DEFAULT, EXTERNAL, INTERNAL, INTERNAL1V1, INTERNAL2V56, LED_BUILTIN, 
                LED_BUILTIN_RX, LED_BUILTIN_TX, DIGITAL_MESSAGE, FIRMATA_STRING, 
                ANALOG_MESSAGE, REPORT_DIGITAL, REPORT_ANALOG, SET_PIN_MODE, 
                SYSTEM_RESET, SYSEX_START, auto, int8_t, int16_t, int32_t, int64_t, 
                uint8_t, uint16_t, uint32_t, uint64_t, char16_t, char32_t, operator, 
                enum, delete, bool, boolean, byte, char, const, false, float, double, 
                null, NULL, int, long, new, private, protected, public, short, 
                signed, static, volatile, String, void, true, unsigned, word, array, 
                sizeof, dynamic_cast, typedef, const_cast, struct, static_cast, union, 
                friend, extern, class, reinterpret_cast, register, explicit, inline, 
                _Bool, complex, _Complex, _Imaginary, atomic_bool, atomic_char, 
                atomic_schar, atomic_uchar, atomic_short, atomic_ushort, atomic_int, 
                atomic_uint, atomic_long, atomic_ulong, atomic_llong, atomic_ullong, 
                virtual, PROGMEM,
                },  
% 
%
  %%% Keyword Color Group 3 %%%  (called KEYWORD1 by arduino)
  keywordstyle=[3]\bfseries\color{arduinoOrange},
  keywords=[3]{  % add built-in functions as a 3rd group
                Serial, Serial1, Serial2, Serial3, SerialUSB, Keyboard, Mouse,
                },      
%
%
  %%% Keyword Color Group 4 %%%  (called KEYWORD2 by arduino)
  keywordstyle=[4]\color{arduinoOrange},
  keywords=[4]{  % add more built-in functions as a 4th group
                abs, acos, asin, atan, atan2, ceil, constrain, cos, degrees, exp, 
                floor, log, map, max, min, radians, random, randomSeed, round, sin, 
                sq, sqrt, tan, pow, bitRead, bitWrite, bitSet, bitClear, bit, 
                highByte, lowByte, analogReference, analogRead, 
                analogReadResolution, analogWrite, analogWriteResolution, 
                attachInterrupt, detachInterrupt, digitalPinToInterrupt, delay, 
                delayMicroseconds, digitalWrite, digitalRead, interrupts, millis, 
                micros, noInterrupts, noTone, pinMode, pulseIn, pulseInLong, shiftIn, 
                shiftOut, tone, yield, Stream, begin, end, peek, read, print, 
                println, available, availableForWrite, flush, setTimeout, find, 
                findUntil, parseInt, parseFloat, readBytes, readBytesUntil, readString, 
                readStringUntil, trim, toUpperCase, toLowerCase, charAt, compareTo, 
                concat, endsWith, startsWith, equals, equalsIgnoreCase, getBytes, 
                indexOf, lastIndexOf, length, replace, setCharAt, substring, 
                toCharArray, toInt, press, release, releaseAll, accept, click, move, 
                isPressed, isAlphaNumeric, isAlpha, isAscii, isWhitespace, isControl, 
                isDigit, isGraph, isLowerCase, isPrintable, isPunct, isSpace, 
                isUpperCase, isHexadecimalDigit, 
                },      
%
%
  %%% Set Other Colors %%%
  stringstyle=\color{arduinoDarkBlue},    
  commentstyle=\color{arduinoGrey},    
%          
%   
  %%%% Line Numbering %%%%
  numbers=left,                    
  numbersep=5pt,                   
  numberstyle=\color{arduinoGrey},    
  %stepnumber=2,                      % show every 2 line numbers
%
%
  %%%% Code Box Style %%%%
  breaklines=true,                    % wordwrapping
  tabsize=2,         
  basicstyle=\ttfamily  
}
                                             %%%
%%% somewhere before \begin{document} in your latex file.                      %%%
%%%                                                                            %%%
%%% In your document, place your arduino code between:                         %%%
%%%   \begin{lstlisting}[language=Arduino]                                     %%%
%%% and:                                                                       %%%
%%%   \end{lstlisting}                                                         %%%
%%%                                                                            %%%
%%% Or create your own style to add non-built-in functions and variables.      %%%
%%%                                                                            %%%
 %%%%%%%%%%%%%%%%%%%%%%%%%%%%%%%%%%%%%%%%%%%%%%%%%%%%%%%%%%%%%%%%%%%%%%%%%%%%%%%% 

\usepackage{color}
\usepackage{listings}    
\usepackage{courier}

%%% Define Custom IDE Colors %%%
\definecolor{arduinoGreen}    {rgb} {0.17, 0.43, 0.01}
\definecolor{arduinoGrey}     {rgb} {0.47, 0.47, 0.33}
\definecolor{arduinoOrange}   {rgb} {0.8 , 0.4 , 0   }
\definecolor{arduinoBlue}     {rgb} {0.01, 0.61, 0.98}
\definecolor{arduinoDarkBlue} {rgb} {0.0 , 0.2 , 0.5 }

%%% Define Arduino Language %%%
\lstdefinelanguage{Arduino}{
  language=C++, % begin with default C++ settings 
%
%
  %%% Keyword Color Group 1 %%%  (called KEYWORD3 by arduino)
  keywordstyle=\color{arduinoGreen},   
  deletekeywords={  % remove all arduino keywords that might be in c++
                break, case, override, final, continue, default, do, else, for, 
                if, return, goto, switch, throw, try, while, setup, loop, export, 
                not, or, and, xor, include, define, elif, else, error, if, ifdef, 
                ifndef, pragma, warning,
                HIGH, LOW, INPUT, INPUT_PULLUP, OUTPUT, DEC, BIN, HEX, OCT, PI, 
                HALF_PI, TWO_PI, LSBFIRST, MSBFIRST, CHANGE, FALLING, RISING, 
                DEFAULT, EXTERNAL, INTERNAL, INTERNAL1V1, INTERNAL2V56, LED_BUILTIN, 
                LED_BUILTIN_RX, LED_BUILTIN_TX, DIGITAL_MESSAGE, FIRMATA_STRING, 
                ANALOG_MESSAGE, REPORT_DIGITAL, REPORT_ANALOG, SET_PIN_MODE, 
                SYSTEM_RESET, SYSEX_START, auto, int8_t, int16_t, int32_t, int64_t, 
                uint8_t, uint16_t, uint32_t, uint64_t, char16_t, char32_t, operator, 
                enum, delete, bool, boolean, byte, char, const, false, float, double, 
                null, NULL, int, long, new, private, protected, public, short, 
                signed, static, volatile, String, void, true, unsigned, word, array, 
                sizeof, dynamic_cast, typedef, const_cast, struct, static_cast, union, 
                friend, extern, class, reinterpret_cast, register, explicit, inline, 
                _Bool, complex, _Complex, _Imaginary, atomic_bool, atomic_char, 
                atomic_schar, atomic_uchar, atomic_short, atomic_ushort, atomic_int, 
                atomic_uint, atomic_long, atomic_ulong, atomic_llong, atomic_ullong, 
                virtual, PROGMEM,
                Serial, Serial1, Serial2, Serial3, SerialUSB, Keyboard, Mouse,
                abs, acos, asin, atan, atan2, ceil, constrain, cos, degrees, exp, 
                floor, log, map, max, min, radians, random, randomSeed, round, sin, 
                sq, sqrt, tan, pow, bitRead, bitWrite, bitSet, bitClear, bit, 
                highByte, lowByte, analogReference, analogRead, 
                analogReadResolution, analogWrite, analogWriteResolution, 
                attachInterrupt, detachInterrupt, digitalPinToInterrupt, delay, 
                delayMicroseconds, digitalWrite, digitalRead, interrupts, millis, 
                micros, noInterrupts, noTone, pinMode, pulseIn, pulseInLong, shiftIn, 
                shiftOut, tone, yield, Stream, begin, end, peek, read, print, 
                println, available, availableForWrite, flush, setTimeout, find, 
                findUntil, parseInt, parseFloat, readBytes, readBytesUntil, readString, 
                readStringUntil, trim, toUpperCase, toLowerCase, charAt, compareTo, 
                concat, endsWith, startsWith, equals, equalsIgnoreCase, getBytes, 
                indexOf, lastIndexOf, length, replace, setCharAt, substring, 
                toCharArray, toInt, press, release, releaseAll, accept, click, move, 
                isPressed, isAlphaNumeric, isAlpha, isAscii, isWhitespace, isControl, 
                isDigit, isGraph, isLowerCase, isPrintable, isPunct, isSpace, 
                isUpperCase, isHexadecimalDigit, 
                }, 
  morekeywords={   % add arduino structures to group 1
                break, case, override, final, continue, default, do, else, for, 
                if, return, goto, switch, throw, try, while, setup, loop, export, 
                not, or, and, xor, include, define, elif, else, error, if, ifdef, 
                ifndef, pragma, warning,
                }, 
% 
%
  %%% Keyword Color Group 2 %%%  (called LITERAL1 by arduino)
  keywordstyle=[2]\color{arduinoBlue},   
  keywords=[2]{   % add variables and dataTypes as 2nd group  
                HIGH, LOW, INPUT, INPUT_PULLUP, OUTPUT, DEC, BIN, HEX, OCT, PI, 
                HALF_PI, TWO_PI, LSBFIRST, MSBFIRST, CHANGE, FALLING, RISING, 
                DEFAULT, EXTERNAL, INTERNAL, INTERNAL1V1, INTERNAL2V56, LED_BUILTIN, 
                LED_BUILTIN_RX, LED_BUILTIN_TX, DIGITAL_MESSAGE, FIRMATA_STRING, 
                ANALOG_MESSAGE, REPORT_DIGITAL, REPORT_ANALOG, SET_PIN_MODE, 
                SYSTEM_RESET, SYSEX_START, auto, int8_t, int16_t, int32_t, int64_t, 
                uint8_t, uint16_t, uint32_t, uint64_t, char16_t, char32_t, operator, 
                enum, delete, bool, boolean, byte, char, const, false, float, double, 
                null, NULL, int, long, new, private, protected, public, short, 
                signed, static, volatile, String, void, true, unsigned, word, array, 
                sizeof, dynamic_cast, typedef, const_cast, struct, static_cast, union, 
                friend, extern, class, reinterpret_cast, register, explicit, inline, 
                _Bool, complex, _Complex, _Imaginary, atomic_bool, atomic_char, 
                atomic_schar, atomic_uchar, atomic_short, atomic_ushort, atomic_int, 
                atomic_uint, atomic_long, atomic_ulong, atomic_llong, atomic_ullong, 
                virtual, PROGMEM,
                },  
% 
%
  %%% Keyword Color Group 3 %%%  (called KEYWORD1 by arduino)
  keywordstyle=[3]\bfseries\color{arduinoOrange},
  keywords=[3]{  % add built-in functions as a 3rd group
                Serial, Serial1, Serial2, Serial3, SerialUSB, Keyboard, Mouse,
                },      
%
%
  %%% Keyword Color Group 4 %%%  (called KEYWORD2 by arduino)
  keywordstyle=[4]\color{arduinoOrange},
  keywords=[4]{  % add more built-in functions as a 4th group
                abs, acos, asin, atan, atan2, ceil, constrain, cos, degrees, exp, 
                floor, log, map, max, min, radians, random, randomSeed, round, sin, 
                sq, sqrt, tan, pow, bitRead, bitWrite, bitSet, bitClear, bit, 
                highByte, lowByte, analogReference, analogRead, 
                analogReadResolution, analogWrite, analogWriteResolution, 
                attachInterrupt, detachInterrupt, digitalPinToInterrupt, delay, 
                delayMicroseconds, digitalWrite, digitalRead, interrupts, millis, 
                micros, noInterrupts, noTone, pinMode, pulseIn, pulseInLong, shiftIn, 
                shiftOut, tone, yield, Stream, begin, end, peek, read, print, 
                println, available, availableForWrite, flush, setTimeout, find, 
                findUntil, parseInt, parseFloat, readBytes, readBytesUntil, readString, 
                readStringUntil, trim, toUpperCase, toLowerCase, charAt, compareTo, 
                concat, endsWith, startsWith, equals, equalsIgnoreCase, getBytes, 
                indexOf, lastIndexOf, length, replace, setCharAt, substring, 
                toCharArray, toInt, press, release, releaseAll, accept, click, move, 
                isPressed, isAlphaNumeric, isAlpha, isAscii, isWhitespace, isControl, 
                isDigit, isGraph, isLowerCase, isPrintable, isPunct, isSpace, 
                isUpperCase, isHexadecimalDigit, 
                },      
%
%
  %%% Set Other Colors %%%
  stringstyle=\color{arduinoDarkBlue},    
  commentstyle=\color{arduinoGrey},    
%          
%   
  %%%% Line Numbering %%%%
  numbers=left,                    
  numbersep=5pt,                   
  numberstyle=\color{arduinoGrey},    
  %stepnumber=2,                      % show every 2 line numbers
%
%
  %%%% Code Box Style %%%%
  breaklines=true,                    % wordwrapping
  tabsize=2,         
  basicstyle=\ttfamily  
}

%\lstdefinestyle{myArduino}{
  language=Arduino,
  %% Add other words needing highlighting below %%
  morekeywords=[1]{endif, },                  % [1] -> dark green
  morekeywords=[2]{SS, PIN_VDIFF_IN, ENC_A, ENC_B, SS_POT, SS_DIS, PIN_VCC_IN, SPI_CLOCK_DIV64 },        % [2] -> light blue
  morekeywords=[3]{},          % [3] -> bold orange
  morekeywords=[4]
  {
    Encoder,
    SPI,
    SPI.h,
    Encoder.h,
    setBitOrder,
    transfer,
    calculateCurrent, 
    setResValue,
    configSPI,
    configPins,
    configDisplay,
    setPotLevel,
    getUserInput,
    controlCurrent,
    setClockDivider,
    clearDisplay,
    setDisplayMessage,
    setBrightness,
    sprintf,
    write,
    getCurrent,
  },      % [4] -> orange
  %% The lines below add a nifty box around the code %%
  frame=single,
  %rulesepcolor=\color{arduinoBlue},
}

\lstset{language=C++,
        basicstyle=\ttfamily\footnotesize,
        keywordstyle=\color{blue}\ttfamily,
        stringstyle=\color{red}\ttfamily,
        commentstyle=\color{codegreen}\ttfamily,
        morecomment=[l][\color{magenta}]{\#},
        numbers=left,                    
        numbersep=5pt,
        numberstyle=\tiny\color{darkgray},
        breaklines=true,
        captionpos=b, % sets the caption-position to bottom
        frame=leftline
}
\usepackage[
pdftitle={\docTitle},
pdfsubject={\pdfDescription},
pdfauthor={\name},
pdfkeywords={\pdfKeyWords},
pdfborder={0 0 0},  % Links nicht sichtbar im pdf
colorlinks = false,
pdfpagelabels,
pdfstartview = FitH,
bookmarksopen = true,
bookmarksnumbered = true,
linkcolor = black,
plainpages = false,
hypertexnames = false,
urlcolor = black,
citecolor = black] {hyperref}

%Einstellugnen für das Dokument, zum Beipiel Formatierung für Verzeichnisse

%Abbildungsverzeichnis
\makeatletter\renewcommand*{\@pnumwidth}{4em}\makeatother
\makeatletter\renewcommand*{\@tocrmarg}{5em}\makeatother
\makeatletter\renewcommand*{\@dotsep }{1}\makeatother

% URL in gleicher Schriftart wie Rest
\urlstyle{same}
% Email mit Verlinkung
\newcommand*{\email}{\href{mailto:lukas.lueck@stud.akad.de}{lukas.lueck@stud.akad.de}} 

%kein einrücken nach Abbildung/Tabelle
\setlength{\parindent}{0pt}

% Formatierungen für das Literaturverzeichnis
%--------------------------------------------
\xpretobibmacro{author}{\begingroup\bfseries}{}{}
\xapptobibmacro{author}{\endgroup}
% Doppelpunkt und Leerzeichen nach Autor
\renewcommand*{\labelnamepunct}{\addspace in\addcolon\addspace} % Punkt am Ende entfernen
\renewcommand*{\finentrypunct}{\addspace}
% Komma zwischen einzelnen Einträgen
\renewcommand*{\newunitpunct}{\addcomma\addspace}
% Punkt am Ende von Zitaten entfernen
\renewcommand*{\bibfootnotewrapper}[1]{\bibsentence#1\addspace}
% Zeilenabstand zwischen einzelnen Einträgen
\setlength{\bibitemsep}{0.5\baselineskip}
%Sortierung bei mehreren Autoren im Format: Name, Vorname
\DeclareNameAlias{sortname}{family-given}
%Trennung zwischen einzelnen Namen mit Semikolon
\renewcommand*{\multinamedelim}{\addsemicolon\space}
\renewcommand*{\finalnamedelim}{\addsemicolon\space}
%Komma zwischen Autor und jahr in Fußnote
\renewcommand*{\nameyeardelim}{\addcomma\space}
%URL ohne Vorgestelltes "URL:"
\DeclareFieldFormat{url}{\addspace\url{#1}}
%number in Runden KLammern
\DeclareFieldFormat{number}{\mkbibparens{#1}}
%URL-Trennung
\apptocmd{\UrlBreaks}{\do\f\do\m}{}{}
\setcounter{biburllcpenalty}{9000}% Kleinbuchstaben
\setcounter{biburlucpenalty}{9000}% Großbuchstaben
\setcounter{biburlnumpenalty}{9000}%Zahlen
%keine Kursive Schrift für Titel im Lit.-Verzeichnis
\DeclareFieldFormat*{title}{#1}


%ABS-WErte mit Skalierten Betragsstrichen
\DeclarePairedDelimiter\abs{\lvert}{\rvert}%
\DeclarePairedDelimiter\norm{\lVert}{\rVert}%

% Swap the definition of \abs* and \norm*, so that \abs
% and \norm resizes the size of the brackets, and the 
% starred version does not.
\makeatletter
\let\oldabs\abs
\def\abs{\@ifstar{\oldabs}{\oldabs*}}
%
\let\oldnorm\norm
\def\norm{\@ifstar{\oldnorm}{\oldnorm*}}
\makeatother

%Farbeinstellung für minted, Matlab
\definecolor{bg}{rgb}{0.95,0.95,0.95}

%Einstellung für pdf-Dokumente, die Übernommen werden, auf Überlappung achten!
%\includepdfset{width=\paperwidth, height=\paperheight, pagecommand={\thispagestyle{fancy}}}


\newacronym{gcd}{GCD}{Greatest Common Divisor}
\usepackage{url}

%URl-Definition für URL mit "%" im Link
\urldef\lmDat\url{https://www.ti.com/lit/ds/symlink/lm35.pdf?ts=1682065674820&ref_url=https%253A%252F%252Fwww.ti.com%252Fproduct%252FLM35}

%--------------------------------------------------------------------------------------------------------------------------------
%                                       Dokument begin
%--------------------------------------------------------------------------------------------------------------------------------
\newcounter{savepage} % Zähler für Seitenzahl, für später ab Römischer Nummerierung
\begin{document}
    %--------------------------------------------------------------------------------------------------------------------------------
    %                                       Deckblatt
    %--------------------------------------------------------------------------------------------------------------------------------
    % Festlegung der Seitenränder
\newgeometry{left=20mm, right=20mm, top=30mm, bottom=30mm}
\thispagestyle{empty}
{
    % Zentrieren der gesamten Seite
    \centering
    
    % Abstand von oben
    \vspace*{3cm}
    % Titel der Arbeit
    {\Huge\textbf{\docType}}\\
    \vspace*{2cm}
    {\Huge{\docTitle}}\\
    \vspace*{1cm}
    {\Large{\betreff}}\\
    \vspace*{2cm}

    \begin{table}[H]
        \centering
        \begin{tabular}{ll|ll}
            Name, Vorname:       & \name                        & Studiengang: & \studiFieldOne  \\
            E-Mail:              & \href{mailto:\mail}{\mail}   &              & \studiFieldTwo  \\
                                 &                              & Modul        & \modul          \\
                                 &                              & Abgabe am:   & \finalDate      \\
                                 &                              & Experte      & \dozent                  
        \end{tabular}
    \end{table}
    
    % Abstand von unten
    \vfill    
}






    \newgeometry{left=45mm, right=20mm, top=30mm, bottom=30mm}

    %---------------------------------------------------------------------------------------------------------------------------------
    %                                       Abstract
    %---------------------------------------------------------------------------------------------------------------------------------
    \pagenumbering{Roman}
    \section*{Abstract}
\addcontentsline{toc}{section}{Abstract}
Hier wird das Management Summary leben.
    \newpage

    %--------------------------------------------------------------------------------------------------------------------------------
    %                                       Verzeichnisse
    %--------------------------------------------------------------------------------------------------------------------------------
    \begin{spacing}{1.0} % Verzeichnisse werden mit einzeiligem Abstand gesetzt

        % Inhaltsverzeichnis ohne Nummerierung
        %\pagestyle{ContentHeaderStyle} % Neuen Seitenstil anwenden
        % Inhaltsverzeichnis
        \thispagestyle{empty}
        \pdfbookmark{\contentsname}{toc}
        \tableofcontents 
        \newpage
        
        % ab hier Nummerierung oben in Römisch, beginnend mit 3
        \pagenumbering{Roman}
        \setcounter{page}{3}
        % Abbildungsverzeichnis, 
        \listoffigures 
        \newpage
        
        % Tabellenverzeichnis
        \listoftables 
        \newpage
        
        % Abkürzungsverzeichnis
        \addcontentsline{toc}{section}{Abkürzungsverzeichnis}
        \printacronyms[heading=section*, name=Abkürzungsverzeichnis, template=tabular]
        \newpage
        
        % Formelverzeichnis
        %\listof{Formel}{Formelübersicht}
        \setcounter{savepage}{\value{page}} %speichert Seitenzahl für spätere folgende Römische Nummerierung
    \end{spacing} 
    %-------------------------------------------------------------------------------------------------------------------
    %                     Ab hier der eigentliche Textteil
    %-------------------------------------------------------------------------------------------------------------------
    % ab hier Nummerierung oben in arabisch, beginngend mit 1
    \pagenumbering{Roman}
    \setcounter{page}{4}
    \pagenumbering{arabic}
    \begin{spacing}{1.5} % Zeilenabstand: 1,5 fuer den Textteil
    
        % Einleitung
        \sloppy%Leerräume dürfen gedehnt werden
        \pdfbookmark{Kapitel}{kapitel}
        \section{Einleitung}

\section{Hinführung}
Das ist ein Tes \cite{broy_vorgehen_2021} kjbsdkjafnadpk alsdh lk a. Das ist eine Abkürzung: \ac{api}.
    
        
\subsection{Problemstellung}
    
    
\subsection{Ziel der Arbeit}
    
    
\subsection{Aufbau der Arbeit}
    

        
        % Grundlagen
        \section{Grundlagen}


        
        % Hauppteil
        \include{Text/04_hauptteil}
        
        % Schluss
        \section{Zusammenfassung}
    \subsection{Zusammenfassung der wichtigsten Ergebnisse}
        
    \subsection{Kritische Betrachtung}
        

    \subsection{Ausblicke}
        
    \end{spacing}
    %-------------------------------------------------------------------------------------------------------------------
    %                     Literaturverzeichnis + Anhang
    %-------------------------------------------------------------------------------------------------------------------
    % ab hier Römisch, folgend auf Verzeichnisse
    \pagenumbering{Roman}
    \setcounter{page}{\value{savepage}}
    
    % Anhang, bei Nichtbedarf auskommentieren
    \newgeometry{left=30mm, right=10mm, top=30mm, bottom=20mm}
%Original: \newgeometry{left=45mm, right=20mm, top=30mm, bottom=30mm, landscape}

\section*{Anhang}
\addcontentsline{toc}{section}{Anhang}
\renewcommand{\thesubsection}{\Alph{subsection}}
%Begin Anhang
\subsection{Shared-Memory-API} \label{anhang:sharedMemApi}


\begin{landscape}
\subsection{Fehleruntersuchung}\label{anhang:error_analysis}
    %\input{Tabellen/error_analysis}
\end{landscape}

                    

%[frame=leftline,  fontsize=\scriptsize, breaklines=true, linenos, ]

    






   
    
    % Literaturverzeichnis
    \printbibliography[title=Literaturverzeichnis]
    %\printbibliography[keyword=Bilder, heading=subbibliography, title={Bildernachweis}]
    %\printbibliography[keyword=Teile, heading=subbibliography, title={Nachweis Teileliste}]
\end{document}
